\documentclass[11pt,a4paper,pdftex]{amsart}

\usepackage[psamsfonts]{amssymb}
%\usepackage{amssymb}
\usepackage{amsmath,amsfonts,latexsym}
\usepackage{graphicx}
\usepackage{t1enc}
%\usepackage[latin1]{inputenc}
%\usepackage[english]{babel}
\usepackage{epsfig}
\usepackage{amscd}
\usepackage{verbatim}
\usepackage[active]{srcltx}
\usepackage[latin1]{inputenc}
\usepackage[spanish]{babel}
%\usepackage[english]{babel}
\usepackage{multicol}


\newtheorem{teo}{Teorema}[section]%establece un contador para
%el entorno 'teo' que aparecerá con el nombre 'Teorema' y que
%volverá a empezar cuando cambien de 'chapter'.

\newtheorem{coro}[teo]{Corolario}%vincula las numeraciones de 'coro'
%con las de 'teo'.

\newtheorem{lema}{Lema}[section]
%\newtheorem{lema}[lema]{Lema}


\newtheorem{definition}{Definici\'on}[section]

\newtheorem{conc}{Conclusión}[section]

\newtheorem{prop}{Proposición}[section]
%\newtheorem{prop}[prop]{Propiedad}

\newtheorem{obs}{Observación}[section]

\renewcommand{\thesection}{{}}

%\newtheorem{obs}{Observación} %numera las 'Observaciones' de corrido
%sin volver a resetear.

\newtheorem{ax}{Axioma}[section]

%\newtheorem{defini}{Definición}[section]

\newtheorem{ej}{Ejercicio}%[section] %numera os 'Ejercicios' reseteando
%en cada 'capítulo'.


%\numberwithin{equation}{section}%numera las formulas reseteando
%cada vez que cambia de 'capítulo'.

% proof
%\newenvironment{proof}
% {\medskip\noindent {\sc Demostración}.-- \ }
% {\hfill\vbox{\hrule height 5pt width 5pt } \bigskip}

%\newcommand{\cqfd}{\hfill\vbox{\hrule height 5pt width 5pt }\bigskip}


\newcommand{\bl}{\begin{lema}}
\newcommand{\el}{\end{lema}}
\newcommand{\bcon}{\begin{conc}}
\newcommand{\econ}{\end{conc}}
\newcommand{\bteo}{\begin{teo}}
\newcommand{\eteo}{\end{teo}}
\newcommand{\bp}{\begin{prop}}
%\newcommand{\ep}{\end{prop}}
\newcommand{\bo}{\begin{obs}}
\newcommand{\eo}{\end{obs}}
\newcommand{\bco}{\begin{coro}}
\newcommand{\eco}{\end{coro}}
\newcommand{\bpf}{\begin{proof}}
\newcommand{\epf}{\end{proof}}
\newcommand{\bax}{\begin{ax}}
\newcommand{\eax}{\end{ax}}
\newcommand{\bdefi}{\begin{defini}}
\newcommand{\edefi}{\end{defini}}
\newcommand{\bej}[1]{\begin{ej}\rm{#1}}
\newcommand{\eej}{\end{ej}\vspace{-0.2cm}}

%---------------------------------------

%comandos
\newcommand{\be}{\begin{enumerate}}
\newcommand{\ee}{\end{enumerate}}
\newcommand{\bit}{\begin{itemize}}
\newcommand{\eit}{\end{itemize}}
\newcommand{\bc}{\begin{center}}
\newcommand{\ec}{\end{center}}
\newcommand{\ba}{\begin{array}}
\newcommand{\ea}{\end{array}}
\newcommand{\bq}{\begin{quotation}}
\newcommand{\eq}{\end{quotation}}
\newcommand{\beq}{\begin{equation}}
\newcommand{\eeq}{\end{equation}}
\newcommand{\mc}[1]{\mathcal{#1}}
\newcommand{\mb}[1]{\;\mbox{#1}\;}
\newcommand{\su}[1]{\underline{#1}}
\newcommand{\so}[1]{\overline{#1}}
\newcommand{\ang}[1]{\widehat{#1}}
\newcommand{\arc}[1]{\wideparen{#1}}
\newcommand{\cc}{QQ\;}
\renewcommand{\bf}{\textbf}
\newcommand{\comb}[2]{\left(\!\!\!\ba{c}#1\\[1ex]#2 \ea
\!\!\!\right)}
%-----------------------------------------

%conjuntos
\newcommand{\W}{\mathbb W}
\newcommand{\K}{\mathbb K}
\newcommand{\N}{\mathbb N}
\newcommand{\C}{{\mathcal C}}
\newcommand{\Su}{{\mathcal S}}
\newcommand{\Z}{\mathbb Z}
\newcommand{\Q}{\mathbb Q}
\newcommand{\R}{\mathbb R}
\newcommand{\F}{\mathbb F}
\newcommand{\A}{\mathbb A}
\newcommand{\V}{\mathbb V}
\newcommand{\I}{\mathbb I}
\newcommand{\0}{\mathbb O}

%------------------------------------------
%operaciones
\newcommand{\8}{\infty}
\newcommand{\ie}{\langle}
\newcommand{\de}{\rangle}
\newcommand{\pe}[2]{\ie {#1} \, ; \, {#2} \de }
\newcommand{\f}[1]{\overrightarrow{#1}}
\newcommand{\DT}[2]{\frac{d{#1}}{d{#2}}}
\newcommand{\ds}{\displaystyle}
\newcommand{\dg}{\Delta}
\newcommand{\g}{\nabla}
\newcommand{\D}{\mbox{div}}
\newcommand{\Dp}[1]{\partial_{#1}}
\newcommand{\DP}[2]{\frac{\partial{#1}}{\partial{#2}}}
\newcommand{\sen}[1]{\mbox{sen}\;{#1}}
\newcommand{\Lp}[1]{L^{#1}}
\newcommand{\To}{\longrightarrow}
%-------------------------------------
%griegos
\newcommand{\vi}{\varphi}
\newcommand{\om}{\omega}
\newcommand{\Om}{\Omega}
\newcommand{\ve}{\varepsilon}



%-------------------------

%comandos particulares

\newcommand{\id}[1]{\text{id}_{#1}}
\newcommand{\mD}{\mc{D}}
\newcommand{\mC}{\mc{S}}
\newcommand{\mS}{\mc{SS}}
\newcommand{\mH}{\mc{H}}
\newcommand{\HD}{\mc{H}_\mD}

\newcommand{\ep}{\varepsilon}


%%%%%%%%%%%%%%%%%%%%%%%%%%%%%%%%%%%%%%%%%%%%%%%%%%%%%%%%%%%%%%%%%
%Un comando para insertar dibujos
% \newcommand{\putfig}[4]{\bigskip \bigskip
%              \begin{figure}[ht]
%              \epsfxsize=#1cm\hfil{\epsfbox{#2}}
%              \caption{#3}
%          \label{#4}
%              \end{figure}\bigskip}

%La sintaxis
%\putfig{ancho}{.eps}{titulo}{label}

%%%%%%%%%%%%%%%%%%%%%%%%%%%%%%%%%%%%%%%%%%%%%%%%%%%%%%%


%Un comando para insertar dos dibujos
\newcommand{\putfigg}[4]{\bigskip \bigskip
             \begin{figure}[ht]
             \epsfxsize=5cm\hfil{\epsfbox{#1}}
             \epsfxsize=5cm\hfil{\epsfbox{#2}}
             \caption{#3}
         \label{#4}
             \end{figure}\bigskip}
%La sintaxis
%\putfigg{.eps}{.eps}{titulo1}{label1}

%%%%%%%%%%%%%%%%%%%%%%%%%%%%%%
%un dibujo en jpg

% \newcommand{\putjpg}[3]{\bigskip
% \begin{center}
%  \begin{figure}[ht]
%   \hspace{5cm}\includegraphics[scale=1.7]{#1.jpg}
%   \caption{#2}
%   \label{#3}
%   \bigskip
%   \bigskip
%  \end{figure}
% \end{center}
% }

%la sintaxis \putjpg{.jpg}{Titulo}{label}


%----------------------------------------------------
%diseño de la página

\vfuzz7pt % Don't report over-full v-boxes if over-edge is 7pt small
\hfuzz7pt % Don't report over-full h-boxes if over-edge is 7pt small

\pagestyle{myheadings}
%\pagestyle{empty}: elimina números de página y encabezados

%\flushbottom \textwidth17cm \textheight23cm \hoffset=-2cm
%\voffset=-0.5cm

\flushbottom \textwidth17cm \textheight25cm \hoffset=-2cm
\voffset=-2cm

\begin{document}



%\markboth{}{\hfil {\sc An\'alisis II--An\'alisis matem\'atico II--Matem\'atica 3. 1er. Cuatrimestre 2008}}

\begin{center}

\bf{\Large An\'alisis II -- An\'alisis matem\'atico II -- Matem\'atica 3.} \\
\bigskip
\bf{\large Segundo Cuatrimestre de 2025}\\
\bigskip
\bf{Pr\'actica 4 - Teoremas de Stokes y de Gauss. Campos conservativos. Aplicaciones.}
\end{center}

\bigskip\bigskip

\setcounter{equation}{0}

\bej  Verificar el teorema de Stokes para el hemisferio superior $z=\sqrt{%
1-x^2-y^2}$,\ $z\ge 0$, y el campo vectorial  ${\bf F}(x,y,z)=(x,y,z)$.
\eej

\bej  Sea $S$ la superficie cil\'{\i }ndrica con tapa, que es uni\'{o}n de
dos superficies $S_1$ y $S_2$, donde $S_1$ es el conjunto de $(x,y,z)$ con $%
x^2+y^2=1$, \ $0\le z\le 1$ y $S_2$ es el conjunto de $(x,y,z)$ con $%
x^2+y^2+(z-1)^2=1$,\ $z\geq 1$, orientadas con la normal que apunta hacia afuera
del cilindro y de la esfera, respectivamente. Sea ${\bf F}%
(x,y,z)=(zx+z^2y+x,z^3yx+y,z^4x^2)$. Calcular $\int_S(\nabla \times {\bf F}%
)\cdot \,d{\bf S}.$
\eej

\bej\ \

\begin{enumerate}
\item[a).]  Considerar dos superficies $S_1$ y $S_2$ con la misma frontera $%
\partial S$. Describir, mediante dibujos, como deben orientarse $S_1$ y $S_2$
para asegurar que
\[
\int_{S_1}(\nabla \times {\bf F})\cdot \,d{\bf S}=\int_{S_2}(\nabla \times
{\bf F})\cdot \,d{\bf S}
\]

\item[b).]  Deducir que si $S$ es una superficie cerrada, entonces
\[
\int_S(\nabla \times {\bf F})\cdot \,d{\bf S}=0
\]
(una superficie cerrada es aquella que constituye la frontera de una
regi\'{o}n en el espacio; as\'{\i }, por ejemplo, una esfera es una
superficie cerrada).

\item[c).]  Calcular $\int_S(\nabla \times {\bf F})\cdot \,d{\bf S}$, donde $S$
es el elipsoide $x^2+y^2+2z^2=10$, y ${\bf F}=(\sin xy,e^x,-yz)$.
\end{enumerate}
\eej

%\bej  Verificar el teorema de Stokes para la helicoide parametrizada por
%$\Phi (r,\theta
%)=(r\cos \theta ,r\sin \theta ,\theta ),$ $(r,\theta )\in [0,1]\times [0,\pi
%/2]$ y el campo vectorial ${\bf F}(x,y,z)=(z,x,y)$
%\eej

\bej  Estudiar la aplicabilidad del teorema de Stokes al campo ${\bf F}%
=(-\frac y{x^2+y^2},\frac x{x^2+y^2},0)$ y la superficie $S$, en cada uno de los siguientes casos:

\begin{enumerate}
\item[a).]  $S=$ c\'{\i }rculo de radio $a>0$ centrado en el origen en el plano $z=0$.

\item[b).]  $S=$ regi\'{o}n del plano $z=0$ entre $x^2+y^2=1$ y $x+y=1$.
\end{enumerate}
\eej


\bej  Evaluar $\int_\C{\bf F}\cdot \,d{\bf s}$, donde

\begin{enumerate}
\item[a).]  ${\bf F}=(2xyz+\mbox{sen\,} x,x^2z,x^2y)$, y $\C$ es la curva que est\'{a}
parametrizada por $({\cos }^5t,{\mbox{sen\,} }^3t,t^4)$, $0\le t\le \pi $.

\item[b).]  ${\bf F}=(\cos xy^2-xy^2\,\mbox{sen\,} xy^2,-2x^2y\,\mbox{sen\,} xy^2,0)$, y $\C$ es
la curva parametrizada
por $(e^t,e^{t+1},0)$, $-1\le t\le 0$.
\end{enumerate}
\eej

\bej Calcular
$$
\int_\C \big(y+\sen x\big)\,dx+\Big(\frac32 z^2+\cos y\Big)\,dy+2x^3\,dz,
$$
donde $\C$ es la curva orientada parametrizada por $\sigma(t)=\big(\sen t,\cos t,\sen 2t\big)$,
$0\le t\le 2\pi$.

Sugerencia: Observar que $\C$ se encuentra en la superficie $z=2xy$.
\eej



\bej Sea $f\in C^1(B)$ donde $B$ es una bola en $\R^3$. Deducir que si $\nabla f=0$
en $B$ se sigue que $f$ es constante en $B$.
\eej

\bej Calcular la integral de
línea $\int_{\mc C}\,\textbf{F}\cdot d\textbf{s}$ donde $\textbf{F}$ es el campo vectorial
definido por: 
$$
\textbf{F}(x,y,z)=(2xy+z^2,x^2-2yz,2xz-y^2)
$$
y ${\mc C}$ es la curva que está contenida en la esfera $x^2+y^2+z^2=1$ 
y el plano de ecuación $y=x$ recorrida desde el punto $\left( \frac{1}{\sqrt{2}}, \frac{1}{\sqrt{2}}, 0 \right)$ 
al polo norte.
\eej



\bej Rehacer el ejercicio 16) de la Pr\'{a}ctica 2, usando el
Teorema de Gauss.
\eej

\bej  Calcular $\int_S(x+y+z)\,dS$ donde $S$ es el borde de la bola unitaria,
es decir
\[
S=\{(x,y,z)/x^2+y^2+z^2=1\}
\]
\eej


%\bej  Verificar el Teorema de la divergencia para ${\bf F}=(x,y,z)$ y $%
%\Omega $ el s\'{o}lido intersecci\'{o}n de $x^2+y^2\le 1$ y $x^2+y^2+z^2\le 4
%$.
%\eej


\bej  Calcular $\int_S{\bf F}\cdot \,d{\bf S}$, siendo ${\bf F}%
=(x^3,y^3,z^3)$ y $S$ la  esfera de radio $R$ con la normal que apunta hacia adentro.

\eej

\bej Sea $\C$ la curva en el plano $xz$ dada en polares por:
\[
r(\varphi )=\frac{4\sqrt{3}}9\left( 2-\cos (2\varphi )\right) \quad \,\text{%
para\ }\frac \pi 6\leq \varphi \leq \frac{5\pi }6,
\]
donde $\varphi \,$ es el \'{a}ngulo que forma el radio vector con el
semieje positivo de las $z$. Sea $S$ la superficie que se obtiene por
\textbf{revoluci\'{o}n} de esta curva alrededor del eje $z$.

\putfigg{dibujo1.eps}{dibujo2.eps}{}{}

En el primer dibujo se muestra la superficie $S,\,$en el segundo se
realiz\'{o} un corte de la misma para que se aprecie mejor su forma.

Calcular el \textbf{flujo} a trav\'{e}s de $S$ en el sentido
``externo'' del campo
\[
F(x,y,z)=(x,y,-2z).
\]
\eej

%\bej  Rehacer el ejercicio \ref{2}) de esta pr\'{a}ctica usando el Teorema
%de Gauss.
%\eej



\bej  Calcular el flujo del campo $F(x,y,z)=(0,0,a^2-x^2-y^2)$ a trav\'{e}s
de las siguientes {\bf secciones oblicuas} del cilindro $x^2+y^2\le a^2:$

\begin{enumerate}
\item[a).]  {\bf Secci\'{o}n oblicua} determinada por la intersecci\'{o}n del
cilindro con el plano de ecuaci\'{o}n $y+z=1,$ de modo que la normal en el
punto $(0,0,1)$ apunte en la direcci\'{o}n $(0,1,1).\medskip $

\item[b).]  {\bf Secci\'{o}n oblicua} determinada por la intersecci\'{o}n del
cilindro con el plano de ecuaci\'{o}n $z=0,$ de modo que la normal en el
punto $(0,0,0)$ apunte en la direcci\'{o}n $(0,0,1).\medskip $
\end{enumerate}

\textquestiondown Depende el flujo del \'{a}rea de la secci\'{o}n?. {\bf %
Justifique.}
\eej

\bej Dada la funci\'{o}n $f\left(  x\right)  =\frac{1}%
{2}xe^{2-2x}$ podemos describir  la superficie de la
calabaza de un mate como la superficie de rotación alrededor del eje $z$ de la curva
$x=f(z)$, $0\le z\le 1$.

Para una idea gr\'{a}fica ver la figura.

\putfigg{dibujo5.eps}{dibujo4.eps}{}{}

Cuando el mate se encuetra cargado de yerba y de agua caliente, el calor es un campo
dado por
\[
F\left(  x,y,z\right)  =\left(  x,y,z-\frac{1}{2}\right)
\]
Calcular el flujo t\'{e}rmico saliente que atraviesa la superficie de la calabaza del mate.
\eej

\bej Sea $\mc S$ la superficie dada por el gráfico de la función
$f(x,y)=\dfrac{1}{1+x^2+y^2}$\, con 
\\ $\|(x,y)\|\leq 1$ y sea
$\textbf{F}(x,y,z)=\Big(\dfrac{zx}{x^2+y^2},\dfrac{zy}{x^2+y^2},0\Big).$ Hallar
$$\iint_{\mc S}\,\textbf{F}\cdot d\textbf{S}.$$

{\em Piense antes de actuar}.
\eej

\bej Se sabe que $\mbox{div\,}\textbf{rot\,G}=0$ para todo campo vectorial
$\textbf{G}\in C^1$. Además, si $\textbf{F}\in C^1(\R^3)$ es tal que $\mbox{div\,}\textbf{F}=0$
en $\R^3$, existe $\textbf{G}\in C^2(\R^3)$ tal que $\textbf{F}=\textbf{rot\,G}$.
Por ejemplo, tomar
$$
\begin{aligned}
&G_1(x,y,z)=\int_0^zF_2(x,y,t)\,dt-\int_0^yF_3(x,t,0)\,dt,\\
&G_2(x,y,z)=-\int_0^zF_1(x,y,t)\,dt,\\
&G_3(x,y,z)=0.
\end{aligned}
$$

Considerar el campo gravitatorio $\textbf{F}=-GmM\frac{\textbf{r}}{r^3}$.
Verificar que $\mbox{div\,} \textbf{F}=0$. \textquestiondown Existe un campo
$\textbf{G}\in C^2(\R^3\setminus\{0\})$ tal que $\textbf{F}=\textbf{rot\,G}$?

\medskip

Sugerencia: Ver Ejercicio 12.
\eej


\bej \textquestiondown Es cada uno de los siguientes campos vectoriales el rotor de
algún otro campo vectorial? De ser así, hallar el campo vectorial.
\begin{enumerate}
\item[a).] $\textbf{F}=(x,y,z)$.
\item[b).] $\textbf{F}=(x^2+1, x-2xy,y)$.
\end{enumerate}
\eej

\bej Para cada $R>0$ sea $S_R=\big\{(x,y,z)\,/\,x^2+y^2+z^2=R^2\,,\,z\ge0\big\}$ orientada con la
normal que apunta hacia arriba, y sea el campo
$$
\textbf{F}(x,y,z)=\big(xz-x\cos z,-yz+y\cos z, 4-x^2-y^2\big).
$$
Determinar $R$ de modo que el flujo del campo $\textbf{F}$ a través de $S_R$ sea máximo.
\eej




%
%\bej Sea $\textbf{V}=\big(\frac{-y}{\sqrt{x^2+y^2}},\frac{x}{\sqrt{x^2+y^2}},0\big)$ el campo de velocidades de un fluido. Encontrar los puntos
%alrededor de los cuales la circulación es máxima.
%\eej

\bej Usando el teorema de Gauss, probar las {\it Identidades de Green}:
\[
\int_{\partial \Omega }f\nabla g\cdot {\bf n}\,dS=\int_\Omega (f\Delta
g+\nabla f\cdot \nabla g)\,dx\,dy\,dz,
\]
\[
\int_{\partial \Omega }(f\nabla g-g\nabla f)\cdot {\bf n}\,dS=\int_\Omega
(f\Delta g-g\Delta f)\,dx\,dy\,dz.
\]

Aqu\'{\i }{\ }${\bf n}$ es la normal exterior al dominio $\Omega \subset
\R^3$,  $f,g$ son de clase $C^2({\Omega})\cap C^1(\overline{\Omega})$ y, para 
una función $u\in C^2(\Omega)$, $\Delta u=u_{xx}+u_{yy}+u_{zz}$.
\eej

\bej Decimos que $\lambda\in\R$ es un autovalor del operador $\Delta$ definido en el
Ejercicio 25 en $\Omega$ si existe una función $f\in C^2(\Omega)\cap C^1(\overline{\Omega})$
con $f=0$ en $\partial \Omega$, $f\not\equiv0$ tal que $\Delta f=\lambda f$ en
$\Omega$. En ese caso decimos que $f$ es una autofunción asociada a $\lambda$.

 Utilizando las identidades de Green del Ejercicio 25, mostrar que si $\lambda\neq\mu$ son
autovalores de $\Delta$ en $\Omega$ y $f$ y $g$ son autofunciones asociadas
a $\lambda$ y $\mu$ respectivamente se tiene
$$\iiint_\Omega f\,g\,dV=0
$$
\eej

\bej Sea $B$ una bola en $\mathbb{R}^3$. Ver que no puede haber una función $f\not\equiv0$, $f\in C^2(B)\cap C^1(\overline{B})$
que satisfaga
$$
\Delta f=0\quad\mbox{en }B,\qquad f=0\quad\mbox{en }\partial B.
$$

\medskip

Sugerencia: Utilizar las identidades de Green del Ejercicio 25 para deducir que $\nabla f=0$ en $B$.
A continuación utilizar el Ejercicio 9 para deducir que $f$ es constante.
\eej


\end{document}

