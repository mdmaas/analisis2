\documentclass[11pt,a4paper,pdftex]{amsart}
\usepackage[psamsfonts]{amssymb}
%\usepackage{amssymb}
\usepackage{amsmath,amsfonts,latexsym}
\usepackage{t1enc}
\usepackage[english]{babel}
\usepackage{graphicx}
\usepackage{amscd}
\usepackage{verbatim}
\usepackage{multicol}
\usepackage[utf8]{inputenc}

\newtheorem{ej}{Ejercicio}%[section] %numera os 'Ejercicios' reseteando
%en cada 'cap�tulo'.


\numberwithin{equation}{section}%numera las formulas reseteando
%cada vez que cambia de 'cap�tulo'.

\newcommand{\bej}[1]{\begin{ej}\rm{#1}}
\newcommand{\eej}{\end{ej}\vspace{-0.2cm}}
%---------------------------------------
\newcommand{\be}{\begin{enumerate}}
\newcommand{\ee}{\end{enumerate}}
\newcommand{\bit}{\begin{itemize}}
\newcommand{\eit}{\end{itemize}}
\newcommand{\bc}{\begin{center}}
\newcommand{\ec}{\end{center}}
\newcommand{\ba}{\begin{array}}
\newcommand{\ea}{\end{array}}
\newcommand{\bq}{\begin{quotation}}
\newcommand{\eq}{\end{quotation}}
\newcommand{\beq}{\begin{equation}}
\newcommand{\eeq}{\end{equation}}
\newcommand{\mc}[1]{\mathcal{#1}}
\newcommand{\mb}[1]{\;\mbox{#1}\;}
\newcommand{\su}[1]{\underline{#1}}
\newcommand{\so}[1]{\overline{#1}}
\newcommand{\ang}[1]{\widehat{#1}}
\newcommand{\arc}[1]{\wideparen{#1}}
\newcommand{\cc}{QQ\;}
\renewcommand{\bf}{\textbf}
\newcommand{\comb}[2]{\left(\!\!\!\ba{c}#1\\[1ex]#2 \ea
\!\!\!\right)}
%-----------------------------------------
%conjuntos
\newcommand{\W}{\mathbb W}
\newcommand{\K}{\mathbb K}
\newcommand{\N}{\mathbb N}
\newcommand{\C}{\mathcal C}
\newcommand{\Z}{\mathbb Z}
\newcommand{\Q}{\mathbb Q}
\newcommand{\R}{\mathbb R}
\newcommand{\F}{\mathbb F}
\newcommand{\A}{\mathbb A}
\newcommand{\V}{\mathbb V}
\newcommand{\I}{\mathbb I}
\newcommand{\0}{\mathbb O}
%------------------------------------------
%operaciones
\newcommand{\8}{\infty}
\newcommand{\ie}{\langle}
\newcommand{\de}{\rangle}
\newcommand{\pe}[2]{\ie {#1} \, ; \, {#2} \de }
\newcommand{\f}[1]{\overrightarrow{#1}}
\newcommand{\DT}[2]{\frac{d{#1}}{d{#2}}}
\newcommand{\ds}{\displaystyle}
\newcommand{\dg}{\Delta}
\newcommand{\g}{\nabla}
\newcommand{\D}{\mbox{div}}
\newcommand{\Dp}[1]{\partial_{#1}}
\newcommand{\DP}[2]{\frac{\partial{#1}}{\partial{#2}}}
\newcommand{\sen}[1]{\mbox{sen}\;{#1}}
\newcommand{\Lp}[1]{L^{#1}}
\newcommand{\To}{\longrightarrow}
%-------------------------------------
%griegos
\newcommand{\vi}{\varphi}
\newcommand{\om}{\omega}
\newcommand{\Om}{\Omega}
\newcommand{\ve}{\varepsilon}
%-------------------------
%comandos particulares
\newcommand{\id}[1]{\text{id}_{#1}}
\newcommand{\mD}{\mc{D}}
\newcommand{\mC}{\mc{S}}
\newcommand{\mS}{\mc{SS}}
\newcommand{\mH}{\mc{H}}
\newcommand{\HD}{\mc{H}_\mD}
%%%%%%%%%%%%%%%%%%%%%%%%%%%%%%%%%%%%%%%%%%%%%%%%%%%%%%%%%%%%%%%%%
%Un comando para insertar dibujos
\newcommand{\putfig}[4]{\bigskip \bigskip
             \begin{figure}[ht]
             \epsfxsize=#1cm\hfil{\epsfbox{#2}}
             \caption{#3}
         \label{#4}
             \end{figure}\bigskip}
%La sintaxis
%\putfig{ancho}{.eps}{titulo}{label}
%%%%%%%%%%%%%%%%%%%%%%%%%%%%%%%%%%%%%%%%%%%%%%%%%%%%%%%
%Un comando para insertar dos dibujos
\newcommand{\putfigg}[4]{\bigskip \bigskip
             \begin{figure}[ht]
             \epsfxsize=5cm\hfil{\epsfbox{#1}}
             \epsfxsize=5cm\hfil{\epsfbox{#2}}
             \caption{#3}
         \label{#4}
             \end{figure}\bigskip}
%La sintaxis
%\putfigg{.eps}{.eps}{titulo1}{label1}
%%%%%%%%%%%%%%%%%%%%%%%%%%%%%%
%un dibujo en jpg
\newcommand{\putjpg}[3]{\bigskip
\begin{center}
 \begin{figure}[ht]
  \hspace{5cm}\includegraphics[scale=1.7]{#1.jpg}
  \caption{#2}
  \label{#3}
  \bigskip
  \bigskip
 \end{figure}
\end{center}
}
%la sintaxis \putjpg{.jpg}{Titulo}{label}
%----------------------------------------------------
%diseñoo de la página

\vfuzz7pt % Don't report over-full v-boxes if over-edge is 7pt small
\hfuzz7pt % Don't report over-full h-boxes if over-edge is 7pt small

\pagestyle{myheadings}

\renewcommand{\labelenumi}{({\it \alph{enumi}})}
\renewcommand{\labelenumii}{\arabic{enumii})}

\flushbottom \textwidth17cm \textheight23cm \hoffset=-2cm
\voffset=-0.5cm

\begin{document}

\begin{center}
\bf{\Large An\'alisis II -- An\'alisis matem\'atico II -- Matem\'atica 3.} \\
\bigskip
\bf{\large Segundo Cuatrimestre de 2025}\\
\bigskip
\bf{Pr\'actica 3 - Teorema de Green.}
\end{center}

\bigskip\bigskip

\setcounter{equation}{0}

\bej
Verificar el Teorema de Green para el disco $D$ con centro $(0,0)$ y
radio $R$ y las siguientes funciones:
\begin{enumerate}
\item[a)]  $P(x,y)=xy^2,\quad Q(x,y)=-yx^2.$
\item[b)]  $P(x,y)=2y,\quad Q(x,y)=x.$
\end{enumerate}
\eej

\bej
Verificar el Teorema de Green y calcular
$$\int_\C y^2\,dx+x\,dy,$$
siendo $\C$ la curva recorrida en sentido positivo:
\begin{enumerate}
\item[a)]  Cuadrado con vértices $(0,0),\,(2,0),\,(2,2),\,(0,2).$
\item[b)]  Elipse dada por $\frac{x^2}{a^2}+\frac{y^2}{b^2}=1.$
\item[c)]   $\C=\C_1\cup \C_2$, donde $\C_1:\,y=x,\;x\in [0,1]$ y $\C_2:\,y=x^2,\;x\in [0,1].$
\end{enumerate}
\eej

\bej
Usando el teorema de Green, hallar el área de:
\begin{enumerate}
\item[a)]  El disco $D$ con centro $(0,0)$ y radio $R$.
\item[b)]  La región dentro de la elipse $\frac{x^2}{a^2}+\frac{y^2}{b^2}=1.$
\end{enumerate}
\eej

\bej
Sea \(D\) la región encerrada por el eje \(x\) y el arco de cicloide:
\[
x=\theta -\sin \theta,\quad y=1-\cos \theta,\quad 0\le \theta \le 2\pi.
\]
Usando el teorema de Green, calcular el área de \(D\).
\eej

\bej
Hallar el área entre las curvas dadas en coordenadas polares por
\[
\begin{array}{cc}
 r=1+\cos\theta & \textrm{ con } -\pi\le \theta\le \pi,\\ 
 r=\sqrt{\cos^2{\theta}-\sin^2{\theta}} & \text{ con } -\displaystyle \frac{\pi} {4} \le \theta\le \frac{\pi}{4}.
\end{array}
\]
\eej

\bej
Probar la fórmula de integración por partes: Si \(D\subset \R^2\) es un dominio elemental, \(\partial D\) su frontera orientada en sentido antihorario y \({\bf n}=(n_1,n_2)\) la normal exterior a \(D\), entonces
\[
\int_D u\,v_x\,dx\,dy = -\int_D u_x\,v\,dx\,dy + \int_{\partial D} u\,v\,n_1\,ds,
\]
para todo par de funciones \(u,\,v\in C(\bar{D})\cap C^1(D)\).
\eej

\bej
Sea \( C \) la curva definida por la ecuación \( x^2 + 3y^2 = 9 \), con \( x \geq 0 \), recorrida en sentido horario.  
Calcular \( \int_C \mathbf{F} \cdot d\mathbf{s} \), siendo  

\[
\mathbf{F}(x,y) =
\left (
\sin(x)y - \frac{y}{x^2 + y^2},
\frac{x}{x^2 + y^2} - \cos(x)
\right ).
\]
\eej

\bej
Sea \(\C\) la curva
\[
\begin{aligned}
&x=0,\quad 0\le y\le4,\\[1ex]
&y=4,\quad 0\le x\le 4,\\[1ex]
&y=x,\quad 0\le x\le 1,\\[1ex]
&y=2-x,\quad 1\le x\le 2,\\[1ex]
&y=x-2,\quad 2\le x\le 3,\\[1ex]
&y=4-x,\quad 2\le x\le 3,\\[1ex]
&y=x,\quad 2\le x\le 4,
\end{aligned}
\]
orientada positivamente. Calcular
\[
\int_\C \frac{y}{(x-1)^2+y^2}\,dx+\frac{1-x}{(x-1)^2+y^2}\,dy.
\]
\eej

\bej
Sea \(D=\{(x,y)\,:\,1\le x^2+y^2\le4,\;x\ge0\}\). Calcular
\[
\int_{\partial D} x^2y\,dx-xy^2\,dy.
\]
Como siempre, \(\partial D\) se recorre en sentido directo (sentido contrario a las agujas del reloj).
\eej

\bej
Calcular el trabajo efectuado por el campo de fuerzas \(F(x,y)=(y+3x,2y-x)\) al mover una partícula rodeando una vez la elipse \(4x^2+y^2=4\) en el sentido de las agujas del reloj.
\eej

\bej
Sea \(\textbf{F}(x,y)=\big(P(x,y),Q(x,y)\big)=\left(\frac{y}{x^2+y^2},\frac{-x}{x^2+y^2}\right)\). Calcular
\[
\int_\C \textbf{F}\cdot d\textbf{s},
\]
donde \(\C\) es la circunferencia unitaria centrada en el origen orientada positivamente.
Calcular \(Q_x-P_y\). ¿Se satisface en este caso el Teorema de Green?
\eej

\bej
Calcular
\[
\int_{\mathcal{C}} f_1\,dx + f_2\,dy,
\]
siendo
\[
f_1(x,y)=\frac{ x\,\sin\left(\frac{\pi}{2(x^2+y^2)}\right)-y(x^2+y^2)}{(x^2+y^2)^2},\quad
f_2(x,y)=\frac{ y\,\sin\left(\frac{\pi}{2(x^2+y^2)}\right)+x(x^2+y^2)}{(x^2+y^2)^2},
\]
y
\[
\mathcal{C}=\begin{cases}
y=x+1, & \textrm{si } -1\le x \le 0,\\[1ex]
y=1-x, & \textrm{si } 0\le x \le 1,
\end{cases}
\]
recorrida desde \((-1,0)\) hasta \((1,0)\).
\eej

\bej
Determinar todas las circunferencias \(\C\) en el plano \(\R^2\) sobre las cuales vale la siguiente igualdad
\[
\int_\C -y^2 \, dx + 3x \, dy = 6\pi.
\]
\eej

\bej
Calcular la integral
\[
\int_\C \textbf{F}\cdot d\textbf{s},
\]
donde
\[
\textbf{F}(x,y)=(y^2e^x+\cos x+(x-y)^2,\;2ye^x+\sen y),
\]
y \(\C\) es la curva
\[
x^2+y^2=1,\quad y\ge 0,
\]
orientada de manera que comience en \((1,0)\) y termine en \((-1,0)\).
\eej

\bej
Sean \(u,\,v \in C^{1}(D)\), donde
\[
D=\left\{(x,y)\in \R^2 : \frac{x^2}{9}+\frac{y^2}{4} \leq1\right\}.
\]
Consideremos los campos definidos por
\[
\textbf{F}(x,y)=(u(x,y),v(x,y)),\quad \textbf{G}(x,y)=(v_{x} -v_{y},\,u_{x} -u_{y}).
\]
Calcular
\[
\iint\limits_{D} (\textbf{F} \cdot \textbf{G})(x,y) \,dx\,dy,
\]
sabiendo que sobre el borde de \(D\) se tiene \(u(x,y)=x\) y \(v(x,y)=1\).
\eej

\bej
Probar el teorema de la divergencia para el plano: Si $D \subseteq \mathbb{R}^2$ es un dominio elemental, $\partial D$ su frontera orientada y $\textbf{n}$ su normal exterior, entonces:
\begin{equation*}
    \int\int_{D} \operatorname{Div}(F)dxdy = \int_{\partial D} F \cdot \textbf{n} \ dS,
\end{equation*}
para todo campo $F:\mathbb{R}^2\rightarrow \mathbb{R}^2$ de clase $C^1$.  
\eej

\end{document}
