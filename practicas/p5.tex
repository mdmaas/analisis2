\documentclass[11pt,a4paper,pdftex]{amsart}

\usepackage[psamsfonts]{amssymb}
%\usepackage{amssymb}
\usepackage{amsmath,amsfonts,latexsym}
\usepackage{graphicx}
\usepackage{t1enc}
%\usepackage[latin1]{inputenc}
%\usepackage[english]{babel}
\usepackage{epsfig}
\usepackage{amscd}
\usepackage{verbatim}
\usepackage[active]{srcltx}
\usepackage[latin1]{inputenc}
\usepackage[spanish]{babel}
%\usepackage[english]{babel}
\usepackage{multicol}


\newtheorem{teo}{Teorema}[section]%establece un contador para
%el entorno 'teo' que aparecerá con el nombre 'Teorema' y que
%volverá a empezar cuando cambien de 'chapter'.

\newtheorem{coro}[teo]{Corolario}%vincula las numeraciones de 'coro'
%con las de 'teo'.

\newtheorem{lema}{Lema}[section]
%\newtheorem{lema}[lema]{Lema}


\newtheorem{definition}{Definici\'on}[section]

\newtheorem{conc}{Conclusión}[section]

\newtheorem{prop}{Proposición}[section]
%\newtheorem{prop}[prop]{Propiedad}

\newtheorem{obs}{Observación}[section]

\renewcommand{\thesection}{{}}

%\newtheorem{obs}{Observación} %numera las 'Observaciones' de corrido
%sin volver a resetear.

\newtheorem{ax}{Axioma}[section]

%\newtheorem{defini}{Definición}[section]

\newtheorem{ej}{Ejercicio}%[section] %numera os 'Ejercicios' reseteando
%en cada 'capítulo'.


%\numberwithin{equation}{section}%numera las formulas reseteando
%cada vez que cambia de 'capítulo'.

% proof
%\newenvironment{proof}
% {\medskip\noindent {\sc Demostración}.-- \ }
% {\hfill\vbox{\hrule height 5pt width 5pt } \bigskip}

%\newcommand{\cqfd}{\hfill\vbox{\hrule height 5pt width 5pt }\bigskip}


\newcommand{\bl}{\begin{lema}}
\newcommand{\el}{\end{lema}}
\newcommand{\bcon}{\begin{conc}}
\newcommand{\econ}{\end{conc}}
\newcommand{\bteo}{\begin{teo}}
\newcommand{\eteo}{\end{teo}}
\newcommand{\bp}{\begin{prop}}
%\newcommand{\ep}{\end{prop}}
\newcommand{\bo}{\begin{obs}}
\newcommand{\eo}{\end{obs}}
\newcommand{\bco}{\begin{coro}}
\newcommand{\eco}{\end{coro}}
\newcommand{\bpf}{\begin{proof}}
\newcommand{\epf}{\end{proof}}
\newcommand{\bax}{\begin{ax}}
\newcommand{\eax}{\end{ax}}
\newcommand{\bdefi}{\begin{defini}}
\newcommand{\edefi}{\end{defini}}
\newcommand{\bej}[1]{\begin{ej}\rm{#1}}
\newcommand{\eej}{\end{ej}\vspace{-0.2cm}}

%---------------------------------------

%comandos
\newcommand{\be}{\begin{enumerate}}
\newcommand{\ee}{\end{enumerate}}
\newcommand{\bit}{\begin{itemize}}
\newcommand{\eit}{\end{itemize}}
\newcommand{\bc}{\begin{center}}
\newcommand{\ec}{\end{center}}
\newcommand{\ba}{\begin{array}}
\newcommand{\ea}{\end{array}}
\newcommand{\bq}{\begin{quotation}}
\newcommand{\eq}{\end{quotation}}
\newcommand{\beq}{\begin{equation}}
\newcommand{\eeq}{\end{equation}}
\newcommand{\mc}[1]{\mathcal{#1}}
\newcommand{\mb}[1]{\;\mbox{#1}\;}
\newcommand{\su}[1]{\underline{#1}}
\newcommand{\so}[1]{\overline{#1}}
\newcommand{\ang}[1]{\widehat{#1}}
\newcommand{\arc}[1]{\wideparen{#1}}
\newcommand{\cc}{QQ\;}
\renewcommand{\bf}{\textbf}
\newcommand{\comb}[2]{\left(\!\!\!\ba{c}#1\\[1ex]#2 \ea
\!\!\!\right)}
%-----------------------------------------

%conjuntos
\newcommand{\W}{\mathbb W}
\newcommand{\K}{\mathbb K}
\newcommand{\N}{\mathbb N}
\newcommand{\C}{{\mathcal C}}
\newcommand{\Su}{{\mathcal S}}
\newcommand{\Z}{\mathbb Z}
\newcommand{\Q}{\mathbb Q}
\newcommand{\R}{\mathbb R}
\newcommand{\F}{\mathbb F}
\newcommand{\A}{\mathbb A}
\newcommand{\V}{\mathbb V}
\newcommand{\I}{\mathbb I}
\newcommand{\0}{\mathbb O}
\newcommand{\ii}{í}
\newcommand{\flecha}{\to}

%------------------------------------------
%operaciones
\newcommand{\8}{\infty}
\newcommand{\ie}{\langle}
\newcommand{\de}{\rangle}
\newcommand{\pe}[2]{\ie {#1} \, ; \, {#2} \de }
\newcommand{\f}[1]{\overrightarrow{#1}}
\newcommand{\DT}[2]{\frac{d{#1}}{d{#2}}}
\newcommand{\ds}{\displaystyle}
\newcommand{\dg}{\Delta}
\newcommand{\g}{\nabla}
\newcommand{\D}{\mbox{div}}
\newcommand{\Dp}[1]{\partial_{#1}}
\newcommand{\DP}[2]{\frac{\partial{#1}}{\partial{#2}}}
\newcommand{\sen}[1]{\mbox{sen}\;{#1}}
\newcommand{\Lp}[1]{L^{#1}}
\newcommand{\To}{\longrightarrow}
%-------------------------------------
%griegos
\newcommand{\vi}{\varphi}
\newcommand{\om}{\omega}
\newcommand{\Om}{\Omega}
\newcommand{\ve}{\varepsilon}
\newcommand{\di}{\displaystyle}
\newcommand{\erre}{\R}


%-------------------------

%comandos particulares

\newcommand{\id}[1]{\text{id}_{#1}}
\newcommand{\mD}{\mc{D}}
\newcommand{\mC}{\mc{S}}
\newcommand{\mS}{\mc{SS}}
\newcommand{\mH}{\mc{H}}
\newcommand{\HD}{\mc{H}_\mD}

\newcommand{\ep}{\varepsilon}


%%%%%%%%%%%%%%%%%%%%%%%%%%%%%%%%%%%%%%%%%%%%%%%%%%%%%%%%%%%%%%%%%
%Un comando para insertar dibujos
\newcommand{\putfig}[4]{\bigskip \bigskip
             \begin{figure}[ht]
             \epsfxsize=#1cm\hfil{\epsfbox{#2}}
             \caption{#3}
         \label{#4}
             \end{figure}\bigskip}

%La sintaxis
%\putfig{ancho}{.eps}{titulo}{label}

%%%%%%%%%%%%%%%%%%%%%%%%%%%%%%%%%%%%%%%%%%%%%%%%%%%%%%%


%Un comando para insertar dos dibujos
\newcommand{\putfigg}[4]{\bigskip \bigskip
             \begin{figure}[ht]
             \epsfxsize=5cm\hfil{\epsfbox{#1}}
             \epsfxsize=5cm\hfil{\epsfbox{#2}}
             \caption{#3}
         \label{#4}
             \end{figure}\bigskip}
%La sintaxis
%\putfigg{.eps}{.eps}{titulo1}{label1}

%%%%%%%%%%%%%%%%%%%%%%%%%%%%%%
%un dibujo en jpg

\newcommand{\putjpg}[3]{\bigskip
\begin{center}
 \begin{figure}[ht]
  \hspace{5cm}\includegraphics[scale=1.7]{#1.jpg}
  \caption{#2}
  \label{#3}
  \bigskip
  \bigskip
 \end{figure}
\end{center}
}

%la sintaxis \putjpg{.jpg}{Titulo}{label}


%----------------------------------------------------
%diseño de la página

\vfuzz7pt % Don't report over-full v-boxes if over-edge is 7pt small
\hfuzz7pt % Don't report over-full h-boxes if over-edge is 7pt small

\pagestyle{myheadings}
%\pagestyle{empty}: elimina números de página y encabezados

\flushbottom \textwidth17cm \textheight23cm \hoffset=-2cm
\voffset=-0.5cm


\begin{document}



%\markboth{}{\hfil {\sc An\'alisis II--An\'alisis matem\'atico II--Matem\'atica 3. 1er. Cuatrimestre 2008}}

\begin{center}

\bf{\Large An\'alisis II--An\'alisis matem\'atico II--Matem\'atica 3.} \\
\bigskip
\bf{\large Segundo Cuatrimestre de 2025}\\
\bigskip
\bf{Pr\'actica 5 - Ecuaciones Diferenciales.}
\end{center}

\bigskip\bigskip

\setcounter{equation}{0}
%\bigskip

\noindent \bf {Ecuaciones Diferenciales de 1er. Orden}

\bigskip


\bej Para cada una de las ecuaciones diferenciales que siguen,
encontrar la soluci\'on general y la
soluci\'on  particular que satisfaga la condici\'on dada:
\[
\begin{array}{llll}
\mbox{a)}&x'-2tx=t,\ \, \quad x(1)=0,\qquad
\ \, \mbox{b)}&x'=\di\frac{1+x^2}{1+t^2},\quad x(1)=0,\\
\\
\mbox{c)}&x'=\di\frac{1+x}{1+t}, \qquad x(0)=1,\qquad
\,\mbox{d)}&x'=\di\frac{1+x}{1-t^2}, \quad x(0)=1, \\
\\
\mbox{e)} &x'-x^{1/3}=0,\ \quad x(0)=0, \qquad
\mbox{f)}&x'=\di\frac{1+x}{1+t}, \quad x(0)=-1.
\end{array}
\]

En todos los casos  dar el intervalo maximal de existencia de las soluciones y
decir si son \'unicas. En los casos en que el intervalo maximal de existencia no es
la recta real, analizar cu\'al es la posible causa.
\eej

\bigskip
 \bej Si $y=y(t)$ denota el n\'umero de habitantes
de una poblaci\'on en funci\'on del tiempo, se denomina tasa de
cre\-ci\-miento de la poblaci\'on a la funci\'on definida como el
cociente $y'/y$.

\begin{enumerate}
\item Caracterizar (encontrar la ecuaci\'on) de las poblaciones con
tasa de cre\-ci\-miento cons\-tante.

\item Dibujar el gr\'afico de $y(t)$ para poblaciones con tasa de
crecimiento cons\-tante, positiva y negativa.

\item ?`Cu\'ales son las poblaciones con tasa de crecimiento nula?

\item Una poblaci\'on tiene tasa de crecimiento constante. El 1 de
enero de 2002 ten\ii a 1000 individuos, y cuatro meses despu\'es
ten\ii a 1020. Estimar el n\'umero de individuos que tendr\'a el
1 de enero del a\~no 2022, usando los resultados anteriores.

\item Caracterizar las poblaciones cuya tasa de crecimiento es una
funci\'on lineal de $t$ ($at + b$).

\item Caracterizar las poblaciones cuya tasa de crecimiento es
igual a $r-cy$, donde $r$ y $c$ son constantes positivas. Este es
el llamado crecimiento log\ii stico, en tanto que el
correspondiente a tasas constantes es llamado crecimiento
exponencial (por razones obvias ?`no?). Para poblaciones
peque\~nas, ambas formas de crecimiento son muy similares.
Comprobar esta afirmaci\'on y comprobar tambi\'en que en el
crecimiento log\ii stico $y(t)$ tiende asint\'oticamente a la
recta $y=r/c$.
\end{enumerate}
\eej

\bigskip


\bej Si un cultivo de bacterias crece con un coeficiente de
variaci\'on proporcional a la cantidad existente y se sabe
adem\'as que la poblaci\'on se duplica en 1 hora ?`Cu\'anto
habr\'a aumentado en 2 horas?.
\eej

\bigskip

\bej Verifique que las siguientes ecuaciones son homog\'eneas de
grado cero y
resuelva:
\[
\mbox{a)}\ tx'=x+2t \exp(-x/t)\qquad \mbox{b)}\ txx'=2x^2-t^{2}\qquad
\mbox{c)}\ x'=\di\frac{x+t}{t},\ \ x(1)=0
\]
\eej

\bigskip

\bej Demuestre que la sustituci\'on $y=at+bx+c$ cambia
$x'=f(at+bx+c)$ en una ecuaci\'on con variables separables y
aplique este m\'etodo para resolver las ecuaciones siguientes:
$$ \mbox{a)}\ x'=(x+t)^{2} \qquad\qquad
 \mbox{b)}\ x'=\mbox{sen}^{2}(t-x+1) $$
\eej

%\bigskip
\newpage

\bej\ \ \newline

\begin{enumerate}
\item Si $ae \ne bd$ demuestre que pueden elegirse constantes $h,
k$ de modo que las sustituciones $t=s-h$, $x=y -k$ reducen
la ecuaci\'on:
\[ \frac{dx}{dt}= F \left( \frac{at+bx+c}{dt+ex+f} \right) \]
a una ecuaci\'on homog\'enea.

\item
 Resuelva las ecuaciones:
 \[
 \begin{aligned}&\qquad\qquad\mbox{a) } x'=\frac{2x-t+4}{x+t-1}
 \qquad \qquad\mbox{b) }x'=\frac{x+t+4}{t-x-6}\\
 &\qquad\qquad \mbox{c) }x'=\frac{x+t+4}{x+t-6},\quad{x(0)=2.\mbox{\ \
 \textquestiondown Se satisface $ae\neq bd$
 en este caso?}}
 \end{aligned}
 \]
\end{enumerate}
\eej

\bigskip

\bej Resuelva las siguientes ecuaciones:
\[
\begin{array}{ll}
\mbox{a)} \ (y-x^{3})dx+(x+y^{3})dy =0
&\hskip-1.2cm\mbox{b)} \ \cos x \cos^{2} y\, dx - 2 \sen x \,\sen y
\cos y\, dy=0 \\
\ \\
\mbox{c)} \ (3x^{2}-y^{2})\,dy -2xy\, dx =0
&\hskip-1.2cm\mbox{d)} \ x\, dy= (x^{5}+x^{3}y^{2} +y)\,dx   \\
\ \\
\text{(e)}\:2(x+y)\sen y\,dx+\big(2(x+y)\sen y+\cos y\big)\,dy=0
&\hskip-1.2cm\mbox{f)} \ 3y \,dx+x\, dy=0
\\
\ \\
\text{(g)}\:  \big(1-y(x+y)\text{tan}\,(xy)\big)\,dx+\big(1-x(x+y)\text{tan}\,(xy)\big)\,dy=0.
\end{array}
\]
\eej

\bigskip

\bej Considere la ecuación lineal de primer orden
\begin{equation}\label{lineal}
\tag{*}
y'+p(x)\,y=q(x).
\end{equation}
\begin{enumerate}
\item Busque una función $\mu(x)$ tal que
$$
\mu(x)\big(y'(x)+p(x)\,y(x)\big)=\big(\mu(x)\,y(x)\big)'.
$$

\item Multiplique la ecuación \eqref{lineal} por $\mu$ y halle su solución general.

\noindent $\mu$ se denomina {\it factor integrante}.
\end{enumerate}

\eej

\bigskip

\bej Hallar la ecuaci\'on de una curva tal que la pendiente de la
recta tangente en un punto cualquiera es la mitad de la
pendiente de la recta que une el punto con el origen.
\eej

\bigskip

\bej Hallar la ecuaci\'on de las curvas tales que la normal en un
punto cualquiera pasa por el origen.
\eej
\bigskip

\bej Demostrar que la curva para la cual la pendiente de la
tangente en cualquier punto es proporcional a la abscisa del
punto de contacto es una par\'abola.
\eej
\bigskip

\bej Hallar la ecuación de una curva del primer cuadrante tal que para cada punto
$(x_0,y_0)$ de
la misma, la ordenada al origen de la recta tangente a la curva en $(x_0,y_0)$ sea
$2(x_0+y_0)$.
\eej

 \ \ \newpage
\bej\ \ \newline

\begin{enumerate}
\item Hallar las soluciones de:
\[
\begin{array}{ll}
\mbox{i)}&y'+y= \sen(x), \\
\mbox{ii)}&y'+y=3 \cos(2x).
\end{array}
\]

\item Halle las soluciones de $ y'+y=\sen(x) +3 \cos(2x)  $
cuya gr\'afica pase por el origen (Piense, y no haga cuentas de m\'as).
\end{enumerate}
\eej
\bigskip

\bej Sea la ecuaci\'on no homog\'enea $y'+a(x)y=b(x)$ donde
$a,b:\erre \flecha \erre$ son continuas con per\ii odo $p>0$ y $b
\not \equiv 0$:
\begin{enumerate}
\item Pruebe que una soluci\'on $\Phi$ de esta ecuaci\'on verifica:
\[ \Phi (x+p) = \Phi (x),\, \forall x\in\erre\
\Leftrightarrow \ \ \Phi(0)= \Phi (p). \]
\item Encuentre las soluciones de per\ii odo $2 \pi$ para las ecuaciones:
\[ y'+3y= \cos(x), \qquad  y'+\cos (x) y = \mbox{sen} (2x). \]
\end{enumerate}
\eej
\bigskip

\bej Suponga que el ritmo al que se enfr\ii a un cuerpo caliente
es proporcional a la diferencia de temperatura entre \'el y el
ambiente que lo rodea (ley de enfriamiento de Newton). Un cuerpo
se calienta 110 $^{\circ}$C y se expone al aire libre a una
temperatura de 10 $\mbox{}^{\circ }$C. Al cabo de una hora su
temperatura es de 60 $^{\circ}$C. ?`Cu\'anto tiempo adicional
debe transcurrir para que se enfr\ii e a 30 $^{\circ}$C?
\eej
\bigskip

\bej Se sabe que el Carbono 14 tiene una semivida de 5600 años. Es decir, su cantidad
se reduce a la mitad por desintegración radioactiva en ese lapso de tiempo.

Si en una roca sedimentaria había al formarse un 40\% de Carbono 14 y ahora hay un 2\%
\textquestiondown Cuánto tiempo pasó desde que se depositaron los sedimentos?

Observación: la tasa de cambio del Carbono 14, $\dot x/x$, es constante.
\eej

\bigskip


\bej La ecuaci\'on $y'+P(x) y =Q(x) y^{n}$, que se conoce como la
ecuaci\'on de Bernoulli, es lineal cuando $n=0,1$. Demuestre que
se puede reducir a una ecuaci\'on lineal para cualquier valor de
$n \ne 1$ por el cambio de variable $z=y^{1-n}$, y aplique este
m\'etodo para resolver las ecuaciones siguientes:
\[
\begin{array}{ll}
\mbox{a)}&xy'+y=x^{4}y^{3},\\
\mbox{b)}&xy^{2}y'+y^{3}=x \cos x,\\
\mbox{c)}&x y'-3y=x^{4}.
\end{array}
\]
\eej

\end{document}
