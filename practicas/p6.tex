\documentclass[11pt]{article}
%\documentstyle{report}

% \usepackage[psamsfonts]{amssymb}
\usepackage{amssymb}
\usepackage{amsmath,amsfonts,latexsym}
\usepackage{graphicx}
\usepackage{t1enc}
%\usepackage[latin1]{inputenc}
%\usepackage[english]{babel}
\usepackage{epsfig}
\usepackage{amscd}
\usepackage{verbatim}
\usepackage[active]{srcltx}
\usepackage[latin1]{inputenc}
\usepackage[spanish]{babel}
%\usepackage[english]{babel}
\usepackage{multicol}

\textheight = 23.5cm \textwidth = 15.5cm \topmargin =-.9cm
\oddsidemargin=0cm

\newtheorem{ej}{Ejercicio}%[section] %numera os 'Ejercicios' reseteando
%en cada 'capítulo'.

\newcommand{\ba}{\begin{array}}
\newcommand{\ea}{\end{array}}

\newcommand{\erre}{\mbox{${\bf R}\ $}}
\newcommand{\rdos}{\mbox{${\bf R}^{2}\ $}}
\newcommand{\rtres}{\mbox{${\bf R}^{3}\ $ }}
\newcommand{\rene}{\mbox{${\bf R}^{n}\ $}}

\newcommand{\ii}{\'{\i}}
\newcommand{\flecha}{\mbox{$\rightarrow$}}

\newcommand{\cuno}{\mbox{${\bf C}^{1}\ $}}
\newcommand{\cdos}{\mbox{${\bf C}^{2}\ $}}

\newcommand{\intab}{\int_{a}^{b}}
\newcommand{\intsgma}{\mbox{$\int_{\sigma}$}}

\newcommand{\sgm}{\mbox{$\sigma \ $}}
\newcommand{\sgmt}{\mbox{$\sigma$(t)\ }}
\newcommand{\sgmpt}{\mbox{$\sigma '$(t)\ }}
\newcommand{\sen}{\mbox{sen\,}}
\newcommand{\di}{\displaystyle}

\newcommand{\bej}[1]{\begin{ej}\rm{#1}}
\newcommand{\eej}{\end{ej}\vspace{-0.2cm}}

\hyphenation{lon-gi-tud} \hyphenation{tra-yec-to-ria}
\hyphenation{ve-ri-fi-que ve-ri-fi-car ve-ri-fi-ca-do}
\hyphenation{si-guien-te si-guien-tes se-guir si-guien-do
se-gui-do} \hyphenation{e-cua-cio-nes} \hyphenation{su-po-ner
su-pon-ga-mos su-pon-ga su-pu-sie-ra}

%\includeonly{a2p567t}
%\includeonly{a2p5}

\begin{document}

%\include{a2p567t}
%\include{a2p5}
%\include{a2p6}
%\include{a2p7}

\begin{center}

\bf{\Large An\'alisis II -- An\'alisis matem\'atico II -- Matem\'atica 3.} \\
\bigskip
\bf{\large Segundo Cuatrimestre de 2025}\\
\bigskip
\bf{Pr\'actica 6 - Ecuaciones de 2do. orden y sistemas de 1er. orden.}
\end{center}

\bigskip

\setcounter{equation}{0}



\bej  Encontrar un sistema fundamental de soluciones reales de las siguientes
ecuaciones:
\[
\begin{array}{ll}
\mbox{i)}&y''-8y'+16y=0\\
\mbox{ii)}&y''-2y'+10y=0\\
\mbox{iii)}&y''-y'-2y=0
\end{array}
\]

 En cada uno de los casos anteriores encontrar una soluci\'on exacta
de la ecuaci\'on no homog\'enea correspondiente con t\'ermino
independiente $x, e^x,1 \mbox{ y } e^{-x}$.

\eej

\bej Sean $(a_{1},b_{1})$ y $(a_{2},b_{2})$ dos puntos del plano tales que
$\frac{a_{1}-a_{2}}{\pi}$ no es un n\'umero entero.
\begin{enumerate}
\item Probar que existe exactamente una soluci\'on de la ecuaci\'on
diferencial $y''+y=0$ cuya gr\'afica pasa por esos puntos.

\item ?`Se cumple en alg\'un caso la parte (a) si $a_{1}-a_{2}$ es un m\'ultiplo
entero de $\pi$?

\item Generalizar el resultado de (a) para la ecuaci\'on $y''+k^{2}y=0$. Discutir
tambi\'en el caso $k=0$.
\end{enumerate}
\eej

\bej Hallar todas las soluciones de $y''-y'-2y=0$ y de $y''-y'-2y= e^{-x}$
que verifiquen:
\[
\begin{array}{llllll}
\mbox{i)} &y(0)=0,\ &y'(0)=1\qquad\quad&\mbox{ii)}&y(0)=1,\ \ \ y'(0)=0\\
\mbox{iii)}&y(0)=0,&y'(0)=0&\mbox{iv)}&\lim_{x \rightarrow +\infty} y(x)=0&\\
\mbox{v)}&y(0)=1&&\mbox{vi)}&y'(0)=1&
\end{array}
\]
\eej

\bej En el interior de la Tierra, la fuerza de gravedad es
proporcional a la distancia al centro. Si se perfora un orificio
que atraviese la Tierra pasando por el centro, y se deja caer
una piedra en el orificio, ?`con qu\'e velocidad llegar\'a al centro?.
\eej

\bej La ecuaci\'on $x^{2}y''+pxy'+qy=0$ ($p,q$ constantes) se denomina
ecuaci\'on de Euler.

\begin{enumerate}
\item Demuestre que el cambio de variables $x= \mbox{e}^{t}$ transforma la
ecuaci\'on en una con coeficientes constantes.

\item Aplique (a) para resolver en $\erre_{>0}$ las ecuaciones:
\[
\begin{array}{ll}
\mbox{i)}&x^{2}y''+2xy'-6y=0\\
\mbox{ii)}&x^{2}y''-xy'+y=2x
\end{array}
\]
\end{enumerate}
\eej


\bej Hallar la soluci\'on general de las siguientes ecuaciones, empleando la
soluci\'on dada:
\[
\begin{array}{llll}
\mbox{i)}&xy''+2y'+xy=0, & I=\erre_{>0}, & y_{1}(x)=\frac{\mbox{sen}x}{x}.\\
\mbox{ii)}&xy''-y'-4x^{3}y=0, & I=\erre_{>0}, & y_{1}(x)=\exp (x^{2}).\\
\mbox{iii)}&xy''-y'-4x^{3}y=0, & I=\erre_{<0}, & y_{1}(x)=\exp (x^{2}).\\
\mbox{iv)}&(1-x^{2})y''-2xy'+2y=0, & I=(-\infty,-1), (-1,1), (1, \infty), 
& y_{1}(x)=x.
\end{array}
\]
El último ítem es un caso especial de la ecuaci\'on $(1-x^{2})y''
-2xy'+p(p+1)y=0$ (ecuaci\'on de Legendre), correspondiente al caso $p=1$,
 en los intevalos en que la
ecuaci\'on es normal.
\eej

\bej Sabiendo que $y_{1}(x)=e^{x^2}$ es
soluci\'on de la ecuaci\'on homog\'enea asociada, hallar todas las
soluciones de $xy''-y'-4x^3y=x^3$.

\eej

\bej Probar que las funciones
\[
\begin{array}{cc}
\begin{array}{ll}
\phi _{1}(t)= &\left\{ \begin{array}{ll}
t^{2}&t \le 0\\
0&t\geq 0
\end{array}
\right.
\end{array}
&
\begin{array}{ll}
\quad \text{ y }\quad \phi _{2}(t)= &\left\{ \begin{array}{ll}
0&t \le 0\\
t^{2}&t\geq 0
\end{array}
\right.
\end{array}
\end{array}
\]
son linealmente independientes en \erre pero que $W(\phi_{1}, \phi_{2})(0)=0$.
?`Existe alg\'un sistema lineal normal de orden 2 definido en alg\'un intervalo
$(-\epsilon, \epsilon)$ que admita a $\{ \phi_{1}, \phi_{2}\}$ como base de
soluciones?
\eej


\bej Hallar la soluci\'on general de los siguientes sistemas
$$
\hskip-.2cm\text{(a)}\: \left\{\ba{l} x_1'=-x_2 \\x_2'=2x_1+3x_2 \ea \right.\qquad\qquad\qquad
\hskip.3cm\text{(b)}\: \left\{\ba{l} x_1'=-8x_1-5x_2\\x_2'=10x_1+7x_2\ea\right.
$$
$$
\hskip1cm\text{(c)}\: \left\{\ba{l}  x_1'=-4x_1+3x_2 \\x_2'=-2x_1+x_2\ea\right.\qquad\hskip1.5cm
\text{(d)}\:\left\{\ba{l} x_1'=-x_1+3x_2-3x_3\\   x_2'=-2x_1+x_2\\ x_3'=-2x_1+3x_2-2x_3
\ea\right.
$$

En cada caso, hallar el conjunto de datos iniciales tales que la
soluci\'on correspondiente tienda a 0 cuando $t$ tienda a
$+\infty$. Ídem con $t$ tendiendo a $-\infty$.

\eej

\bej Dos tanques, conectados mediante tubos, contienen cada uno 24 litros de una soluci\'on salina. Al tanque I entra agua pura a raz\'on de 6 litros por minuto y del tanque II sale, al exterior, el agua que contiene a raz\'on de 6 litros por minuto. Adem\'as el l\'iquido se bombea del tanque I al tanque II a
una velocidad de 8 litros por minuto y del tanque II al tanque I a una
velocidad de 2 litros por minuto. Se supone que los tanques se agitan de igual forma constantemente de manera tal que la mezcla sea homog\'enea. Si en un principio hay $x_0$ kg de sal en el tanque I e $y_0$ Kg de sal en el tanque II, determinar la cantidad de sal en cada tanque a tiempo $t>0$. Cu\'al es el l\'imite, cuando $t\to +\infty$, de las respectivas concentraciones de sal en cada tanque.?
\eej

%\bej Inicialmente el tanque I contiene 100 litros de agua salada a una
%concentraci\'on de 1 kg por litro y el tanque II tiene 100 litros
%de agua pura. El l\ii quido se bombea del tanque I al tanque II a
%una velocidad de 1 litro por minuto, y del tanque II al I a una
%velocidad de 2 litros por minuto. Los tanques se agitan
%constantemente. \textquestiondown Cu\'al es la concentraci\'on en el tanque I
%despu\'es de 10 minutos?
%\eej

\bej Hallar la soluci\'on general de los siguientes sistemas

$$\text{(a)}\; \left\{\ba{l}x_1'=x_1-x_2\\ x_2'=x_1+x_2  \ea \right. \qquad
\ \ \text{(b)}\; \left\{\ba{l}x_1'=2x_1-x_2\\ x_2'=4x_1+2x_2  \ea \right. \qquad
$$
$$\ \ \ \text{(c)}\; \left\{\ba{l}x_1'=2x_1+x_2   \\ x_2'=2x_2    \ea \right. \qquad
\text{(d)}\; \left\{\ba{l}x_1'=-5x_1+9x_2\\ x_2'=-4x_1+7x_2  \ea \right. \qquad
$$

\eej

\bej Hallar la soluci\'on general de los siguientes sistemas
$$
\text{(a)}\: \left\{\ba{l}x_1'=-x_2+2\\ x_2'=2x_1+3x_2+t \ea\right.\qquad
\text{(b)} \:\left\{\ba{l} x_1'=2x_1-x_2+e^{2t}\\ x_2'=4x_1+2x_2+4 \ea\right.
$$
\eej

\end{document}
